\documentclass[a4paper]{article}

\usepackage[margin=2cm]{geometry}
\usepackage[utf8]{inputenc}
\usepackage{booktabs}
\usepackage[fleqn]{amsmath}
\usepackage{amsfonts}
\usepackage{amssymb}
\usepackage{float}
\usepackage{color}
\usepackage{caption}
\usepackage{listings}
\usepackage{graphicx}
\usepackage{struktex}

\DeclareMathOperator*{\search}{SEARCH}
\DeclareMathOperator*{\mindSearch}{\forall SEARCH}

\title{”Programozás”\\ beadandó feladat:\\ 9. feladat}
\date{2017-04-18}
\author{Készítette: Bárdosi Bence\\ Neptun-azonosító: VY9NJN\\ E-mail: bardosi.bence@gmail.com}

\renewcommand*\contentsname{Tartalom}
\renewcommand*\figurename{}


\begin{document}
  \pagenumbering{gobble}
  \maketitle
  \newpage

  \pagenumbering{arabic}
  \tableofcontents
  \newpage

  \section{Dokumentáció}
    \subsection{Feladat}
    Madarak életének kutatásával foglalkozó szakemberek n különböző településen m különböző madárfaj előfordulását tanulmányozzák. Egy adott időszakban megszámolták, hogy az egyes településen egy madárfajnak hány egyedével találkoztak.
    Volt-e olyan település, ahol mindegyik madárfaj előfordult?
    \subsection{Specifikáció}
        \begin{align*}
          \mathbf{A=}\, \left ( adat: \mathbb{N}^{n x m} , l: \mathbb{L} \right )
        \end{align*}
        \begin{align*}
          \mathbf{Ef=}\, \left ( adat=adat' \right )
        \end{align*}
        \begin{align*}
          \mathbf{Uf=}\, \left ( Ef \wedge (l,\_)=\search_{i=1}^n mind(i) \right )
        \end{align*}
        \begin{align*}
          \mathbf{ahol}\, mind(i): \mathbb{N} \rightarrow \mathbb{L}
        \end{align*}
        \begin{align*}
          \mathbf{\acute{e}s}\, \forall i \in [1..n]: mind(i)=\mindSearch_{j=1}^{m} adat[i,j]>0
        \end{align*}
    \subsection{Algoritmus}
      A feladatot a lineáris keresés, az alfeladatot az optimista lineáris keresés programozási tételeire vezetjük vissza.
      \begin{tabular}{cc}
          \begin{minipage}{.5\linewidth}
            \begin{tabular}{ccc}
              & & \\
              & Lin. ker. & \\
              \midrule
              Tétel & & Feladat \\
              \midrule
              $m$ & $\leftarrow$ & 1 \\
              $n$ & $\leftarrow$ & n \\
              $ind$ & $\leftarrow$ & $\_$ \\
              $\beta (i)$ & $\leftarrow$ & $mind(i)$ \\
              \midrule
              & & \\
            \end{tabular}
            \begin{struktogramm}(45,25)
  \assign{\(l, i := hamis, 1\)}
  \while{\(\neg l \wedge i\leq n\)}
    \assign{\(l:=mind(i)\)}
    \assign{\(i:=i+1\)}
  \whileend
\end{struktogramm}

          \end{minipage} &

          \begin{minipage}{.5\linewidth}
            \begin{tabular}{ccc}
              & & \\
              & Opt. lin. ker. & \\
              \midrule
              Tétel & & Feladat \\
              \midrule
              $m$ & $\leftarrow$ & 1 \\
              $n$ & $\leftarrow$ & m \\
              $i$ & $\leftarrow$ & $j$ \\
              $\beta (i)$ & $\leftarrow$ & $adat[i,j]>0$ \\
              \midrule
              & &  \\
            \end{tabular}
            \begin{struktogramm}(50,25)[l:=mind(i)]
  \assign{\(l, j := true, 1\)}
  \while{\(l \wedge j\leq m\)}
    \assign{\(l:=adat[i,j]>0\)}
    \assign{\(j:=j+1\)}
  \whileend
\end{struktogramm}

          \end{minipage}
      \end{tabular}
    \subsection{Implementáció}
        \subsubsection{Adattípusok megvalósítása}
        A tervben szereplő mátrixot \texttt{vector<vector<int>>}-ként deklaráljuk. Mivel a vektor 0-tól indexelődik, azért a tervbeli ciklusok indextartományai a \texttt{0..n–1} és a \texttt{0..m–1} intervallumra módosulnak, ahol a \texttt{n}-re \texttt{t.size()} alakban, \texttt{m}-re pedig \texttt{t[i].size()} alakban hivatkozhatunk.
        \subsubsection{Bemenő adatok formája}
        Az adatokat be lehet olvasni egy szöveges állományból vagy meg lehet adni billentyűzetről. A program először megkérdezi az adatbevitel módját, majd a szöveges állományból való olvasást választva bekéri az állomány nevét. A billentyűzetről vezérelt adatbevitelt a program párbeszéd-üzemmódban irányítja, és azt megfelelő adat-ellenőrzésekkel vizsgálja. A szöveges állomány formája kötött, arról feltesszük, hogy helyesen van kitöltve, ezért ezt külön nem ellenőrizzük. Az első sor a városok és a madárfajok számát tartalmazza, szóközökkel vagy tabulátor jelekkel elválasztva. Ezt követően következnek a városonként megfigyelt madárfajok egyedeinek száma kötött sorrendben. A fájlt egy sorvége jel zárja.
      \subsubsection{Program váz}
      A program több állományból áll. A read csomag (\texttt{read.h, read.cpp}) felel az adatok helyes beolvasásáról és azok ellenőrzéséről. A "madar" csomag (\texttt{madar.h, madar.cpp}) felel feladat megoldásáért. Végül a "teszt" csomag (\texttt{catch.hpp, test.h, test.cpp}) az egységteszt megszervezéséért felel. A csomagokban található függvények/állományok feladatai az alábbi táblázatokból olvashatók.
      \begin{table}[H]
        \caption*{Read csomag}
        \begin{tabular}{l|l}
          \toprule
          Függvény & Feladat \\
          \midrule
          read & Megkérdezi a felhasználótól, hogy milyen módon kívánja az adatokat bevinni.\\
          from\_file &  Megnyitja a megadott szöveges állományt,és beolvassa azadatokat.\\
          from\_console & Párbeszédes formában bekéri a felhasználótól az adatokat.\\
          safecin & Ellenőrzi, hogy a kapott adat természetes szám-e. Hiba esetén értesíti a felhasználót.\\
          \bottomrule
        \end{tabular}
      \end{table}
      \begin{table}[H]
        \caption*{Madar csomag}
        \begin{tabular}{l|l}
          \toprule
          Függvény & Feladat \\
          \midrule
          madar & Megadja, hogy az adott mátrixban létezik-e megfelelő tulajdonságú város.\\
          mind &  Megadja, hogy az adott tömbben minden madárfajból előfordul-e legalább 1.\\
          \bottomrule
        \end{tabular}
      \end{table}
      \begin{table}[H]
        \caption*{Test csomag}
        \begin{tabular}{l|l}
          \toprule
          Állomány & Feladat \\
          \midrule
          test.h & Tartalmazza azt a flag-et, amely megszervezi, hogy programfutás, vagy egységteszt forduljon.\\
          test.cpp & Egységteszteket tartalmazó állomány\\
          catch.hpp & Külső könyvtár, mely megszervezi az automatikus egységtesztet.\\
          \bottomrule
        \end{tabular}
      \end{table}
      \includegraphics[width=\textwidth]{fgv}
    \subsection{Tesztelés}
      \subsubsection{A programozási tételekre épülő (szürke doboz) tesztesetek:}
      \begin{table}[H]
        \caption*{Külső programozási tétel (lin.ker)}
        \begin{tabular*}{\textwidth}{ll}
          \toprule
          \textbf{intervallum hossza} szerint: & \\
          1. \textit{nulla} hosszú: & 0 város, 0 madár \\
          & \quad be11.txt - válasz: Hamis \\
          2. \textit{egy} város: & van "0" madár  \\
          & \quad be12.txt - válasz: Hamis \\
          3. \textit{egy} város: & nincs "0" madár \\
          & \quad be13.txt - válasz: Igaz \\
          4. \textit{több} város: &  \\
          & \quad be14.txt - válasz: Igaz \\
          \textbf{intervallum eleje} szerint: & \\
          & Csak az első városban fordult elő az összes madárból \\
          & \quad be15.txt - válasz: Igaz \\
          \textbf{intervallum vége} szerint: & \\
          & Csak az utolsó városban fordult elő az összes madárból \\
          & \quad be16.txt - válasz: Igaz \\
          \textbf{tételre jellemző esetek} szerint: & \\
          1. Több város, több madár & mindegyik városban előfordult mind: \\
          & \quad be17.txt - válasz: Igaz \\
          2. Több város, több madár & egy városban előfordult mind: \\
          & \quad be18.txt - válasz: Igaz \\
          3. Több város, több madár & egy városban sem fordult elő mind: \\
          & \quad be19.txt - válasz: Hamis \\
          \bottomrule
        \end{tabular*}
      \end{table}
      \begin{table}[H]
        \caption*{Belső programozási tétel (opt.lin.ker)}
        \begin{tabular*}{\textwidth}{ll}
          \toprule
          \textbf{intervallum hossza} szerint: & \\
          1. \textit{nulla} hosszú: & 0 madár \\
          & \quad be21.txt - válasz: Igaz \\
          2. \textit{egy} "0" madár & \\
          & \quad be22.txt - válasz: Hamis \\
          3. \textit{egy} nem "0" madár &\\
          & \quad be23.txt - válasz: Igaz \\
          4. \textit{több} madár: &  \\
          & \quad be24.txt - válasz: Igaz \\
          \textbf{intervallum eleje} szerint: & \\
          & Csak az első madár "0" \\
          & \quad be25.txt - válasz: Hamis \\
          \textbf{intervallum vége} szerint: & \\
          & Csak az utolsó madár "0" \\
          & \quad be26.txt - válasz: Hamis \\
          \textbf{tételre jellemző esetek} szerint: & \\
          1. Több madár, mind nem "0": & \\
          & \quad be27.txt - válasz: Igaz \\
          2. Több madár, egy "0": & \\
          & \quad be28.txt - válasz: Hamis \\
          3. Több madár, több 0: & \\
          & \quad be29.txt - válasz: Hamis \\
          \bottomrule
        \end{tabular*}
      \end{table}
    \subsubsection{A megoldó programra épülő (fehér doboz) tesztesetek:}
      \begin{itemize}
        \item Hibás vagy nem létező állománynév megadása.
        \item Menü választás tesztelése.
        \item Beolvasás mindkét módozatána tesztelése.
        \item Hibás adatok beolvasásának tesztelése.
      \end{itemize}
\end{document}
